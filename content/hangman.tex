\section{Example: A ``Mystery Word'' Guessing Game}

\label{sect:hangman}

In this section, we will use the techniques and specific effects discussed
in the tutorial so far to implement a larger example, a simple text-based
word-guessing game. In the game, the computer chooses a word, which the
player must guess letter by letter. After a limited number of wrong
guesses, the player loses\footnote{Readers may recognise this game
by the name ``Hangman.''}.

We will implement the game by following these steps:

\begin{enumerate}
\item Define the game state, in enough detail to express the rules
\item Define the rules of the game (i.e. what actions the player may take, 
and how these actions affect the game state)
\item Implement the rules of the game (i.e. implement state updates for each 
action)
\item Implement a user interface which allows a player to direct actions
\end{enumerate}

\noindent
Step 2 may be achieved by defining an effect which depends on the state defined
in step 1. Then step 3 involves implementing a \texttt{Handler} for this effect.
Finally, step 4 involves implementing a program in \texttt{Eff} using the
newly defined effect (and any others requries to implement the interface).

\subsection{Step 1: Game State}

First, we categorise the game states as running games (where there are a 
number of guesses available, and a number of letters still to guess), or
non-running games (i.e. games which have not been started, or games which
have been won or lost).

\begin{code}
data GState = Running Nat Nat | NotRunning
\end{code}

\noindent
Notice that at this stage, we say nothing about what it means to make a guess,
what the word to be guessed is, how to guess letters, or any other implementation
detail. We are only interested in what is necessary to describe the game
rules.

We will, however, parameterise a concrete game state \texttt{Mystery} over
this data:

\begin{code}
data Mystery : GState -> Type
\end{code}

\subsection{Step 2: Game Rules}

We describe the game rules as a dependent effect, where each action
has a \emph{precondition} (i.e. what the game state must be before carrying
out the action) and a \emph{postcondition} (i.e. how the action affects
the game state). Informally, these actions with the pre- and postconditions are:

\begin{description}
\item[Guess] Guess a letter in the word.
\begin{itemize}
\item Precondition: The game must be running, and there must be 
both guesses still available, and letters still to be guessed.
\item Postcondition: If the guessed letter is in the word and not yet guessed,
reduce the number of letters, otherwise reduce the number of guesses.
\end{itemize}
\item[Won] Declare victory 
\begin{itemize}
\item Precondition: The game must be running, and there must be no letters
still to be guessed.
\item Postcondition: The game is no longer running.
\end{itemize}
\item[Lost] Accept defeat
\begin{itemize}
\item Precondition: The game must be running, and there must be no guesses left.
\item Postcondition: The game is no longer running.
\end{itemize}
\item[NewWord] Set a new word to be guessed 
\begin{itemize}
\item Precondition: The game must not be running.
\item Postcondition: The game is running, with 6 guesses available (the choice
of 6 is somewhat arbitrary here) and the number of unique letters in the word
still to be guessed.
\end{itemize}
\item[StrState] Get a string representation of the game state. This is for
display purposes; there are no pre- or postconditions.
\end{description}

\noindent
We can make these rules precise by declaring them more formally in an
effect signature:

\begin{code}
data MysteryRules : Effect where
     Guess : (x : Char) ->
             { Mystery (Running (S g) (S w)) ==>
               {inword} if inword then Mystery (Running (S g) w)
                                  else Mystery (Running g (S w)) }
                MysteryRules Bool
     Won  : { Mystery (Running g 0) ==> Mystery NotRunning } MysteryRules ()
     Lost : { Mystery (Running 0 g) ==> Mystery NotRunning } MysteryRules ()
     NewWord : (w : String) ->
               { h ==> Mystery (Running 6 (length (letters w))) } MysteryRules
     StrState : { Mystery h } MysteryRules String
\end{code}

\noindent
This description says nothing about how the rules are implemented. In particular,
it does not specify \emph{how} to tell whether a guessed letter was in a word,
just that the result of \texttt{Guess} depends on it.

Nevertheless, we can still create an \texttt{EFFECT} from this, and use it
in an \texttt{Eff} program. Implementing a \texttt{Handler} for
\texttt{MysteryRules} will then allow us to play the game.

\begin{code}
MYSTERY : GState -> EFFECT
MYSTERY h = MkEff (Mystery h) MysteryRules
\end{code}

\subsection{Step 3: Implement Rules}

To \emph{implement} the rules, we begin by giving a concrete definition of
game state:

\begin{code}
data Mystery : GState -> Type where
     Init     : Mystery NotRunning 
     GameWon  : (word : String) -> Mystery NotRunning
     GameLost : (word : String) -> Mystery NotRunning
     MkG      : (word : String) ->
                (guesses : Nat) ->
                (got : List Char) ->
                (missing : Vect m Char) ->
                Mystery (Running guesses m)
\end{code}

\noindent
If a game is \texttt{NotRunning}, that is either because it
has not yet started (\texttt{Init}) or because it is won or lost (\texttt{GameWon}
and \texttt{GameLost}, each of which carry the word so that showing the game
state will reveal the word to the player). Finally, \texttt{MkG} captures a 
running game's state, including the target word, the letters successfully
guessed, and the missing letters. Using a \texttt{Vect} for the missing
letters is convenient since its length is used in the type.

To initialise the state, we implement the following functions: \texttt{letters},
which returns a list of unique letters in a \texttt{String} (ignoring spaces)
and \texttt{initState} which sets up an initial state considered valid as a
postcondition for \texttt{NewWord}.

\begin{code}
letters : String -> List Char
initState : (x : String) -> Mystery (Running 6 (length (letters x)))
\end{code}

\noindent
When checking if a guess is in the vector of missing letters, it is convenient
to return a \emph{proof} that the guess is in the vector, using \texttt{isElem}
below, rather than merely a \texttt{Bool}:

\begin{code}
data IsElem : a -> Vect n a -> Type where
     First : IsElem x (x :: xs)
     Later : IsElem x xs -> IsElem x (y :: xs)

isElem : DecEq a => (x : a) -> (xs : Vect n a) -> Maybe (IsElem x xs)
\end{code}

\noindent
The reason for returning a proof is that we can use it to remove an element
from the correct position in a vector:

\begin{code}
shrink : (xs : Vect (S n) a) -> IsElem x xs -> Vect n a
\end{code}

\noindent
We leave the definitions of \texttt{letters}, \texttt{init}, \texttt{isElem}
and \texttt{shrink} as exercises. Having implemented these, the \texttt{Handler}
implementation for \texttt{MysteryRules} is surprisingly straightforward:

\begin{code}
using (m : Type -> Type)
  instance Handler MysteryRules m where
    handle (MkG w g got []) Won k = k () (GameWon w)
    handle (MkG w Z got m) Lost k = k () (GameLost w)
  
    handle st StrState k = k (show st) st
    handle st (NewWord w) k = k () (initState w)
  
    handle (MkG w (S g) got m) (Guess x) k =
        case isElem x m of
             Nothing => k False (MkG w _ got m)
             (Just p) => k True (MkG w _ (x :: got) (shrink m p))
\end{code}

\noindent
Each case simply involves directly updating the game state in a way which
is consistent with the declared rules. In particular, in \texttt{Guess}, if
the handler claims that the guessed letter is in the word (by passing
\texttt{True} to \texttt{k}), there is no way to update the state in such
a way that the number of missing letters or number of guesses does not
follow the rules.

\subsection{Step 4: Implement Interface}

Having described the rules, and implemented state transitions which follow
those rules as an effect handler, we can now write an interface for the
game which uses the \texttt{MYSTERY} effect:

\begin{code}
game : { [MYSTERY (Running (S g) w), STDIO] ==>
         [MYSTERY NotRunning, STDIO] } Eff IO ()
\end{code}

\noindent
The type indicates that the game must start in a running state, with some
guesses available, and eventually reach a not-running state (i.e. won or lost). 
The only way to achieve this is by correctly following the stated rules.
A possible complete implementation of \texttt{game} is presented in Listing
\ref{mword}. 

Note that the type of \texttt{game} makes no assumption that there
are letters to be guessed in the given word (i.e. it is \texttt{w} rather than
\texttt{S w}). This is because we will be choosing a word at random from a
vector of \texttt{String}s, and at no point have we made it explicit that
those \texttt{String}s are non-empty.

Finally, we need to initialise the game by picking a word at random from a
list of candidates, setting it as the target using \texttt{NewWord}, then
running \texttt{game}:

\begin{code}
runGame : { [MYSTERY NotRunning, RND, SYSTEM, STDIO] } Eff IO ()
runGame = do srand (cast !time)
             let w = index !(rndFin _) words
             NewWord w
             game
             putStrLn !StrState
\end{code}

\noindent
We use the system time (\texttt{time} from the \texttt{SYSTEM} effect; see
Appendix \ref{sect:appendix}) to initialise the random number generator,
then pick a random \texttt{Fin} to index into a list of \texttt{words}.
For example, we could initialise a word list as follows:

\begin{code}
words : ?wtype
words = with Vect ["idris","agda","haskell","miranda",
         "java","javascript","fortran","basic",
         "coffeescript","rust"]
  
wtype = proof search
\end{code}

\noindent
\textbf{Aside:} 
Rather than have to explicitly declare a type with the vector's length, it
is convenient to give a metavariable \texttt{?wtype} and let \Idris{}'s
proof search mechanism find the type. This is a limited form of type
inference, but very useful in practice.

\begin{code}[float=h,frame=single,caption={Mystery Word Game Implementation},label=mword]
game : { [MYSTERY (Running (S g) w), STDIO] ==>
         [MYSTERY NotRunning, STDIO] } Eff IO ()
game {w=Z} = Won
game {w=S _}
     = do putStrLn !StrState
          putStr "Enter guess: "
          let guess = trim !getStr
          case choose (not (guess == "")) of
               (Left p) => processGuess (strHead' guess p)
               (Right p) => do putStrLn "Invalid input!"
                               game
  where
    processGuess : Char -> { [MYSTERY (Running (S g) (S w)), STDIO] ==>
                             [MYSTERY NotRunning, STDIO] }
                           Eff IO ()
    processGuess {g} {w} c
      = case !(Main.Guess c) of
             True => do putStrLn "Good guess!"
                        case w of
                             Z => Won
                             (S k) => game
             False => do putStrLn "No, sorry"
                         case g of
                              Z => Lost
                              (S k) => game
\end{code}

\subsection{Discussion}

Writing the rules separately as an effect, then an implementation which uses
that effect, ensures that the implementation must follow the rules. This has
practical applications in more serious contexts; \texttt{MysteryRules} for
example can be though of as describing a \emph{protocol} that a game
player most follow, or alternative a \emph{precisely-typed API}.

In practice, we wouldn't really expect to write rules first then implement
the game once the rules were complete. Indeed, I didn't do so when constructing
this example! Rather, I wrote down a set of likely rules making 
any assumptions \emph{explicit} in the state transitions for \texttt{MysteryRules}. 
Then, when implementing \texttt{game} at
first, any incorrect assumption was caught as a type error. The following
errors were caught during development:

\begin{itemize}
\item Not realising that allowing \texttt{NewWord} to be an arbitrary string
would mean that \texttt{game} would have to deal with a zero-length word as
a starting state.
\item Forgetting to check whether a game was won before recursively calling
\texttt{processGuess}, thus accidentally continuing a finished game.
\item Accidentally checking the number of missing letters, rather than the
number of remaining guesses, when checking if a game was lost.
\end{itemize}

\noindent
These are, of course, simple errors, but were caught by the type checker before
any testing of the game.


