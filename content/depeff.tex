\section{Dependent Effects}

In the programs we have seen so far, the available effects have remained
constant. Sometimes, however, an operation can \emph{change} the available
effects. The simplest example occurs when we have a state with a dependent
type---adding an element to a vector also changes its type, for example,
since its length is explicit in the type. In this section, we will see how
the \effects{} library supports this. Firstly, we will see how states with
dependent types can be implemented. Secondly, we will see how the effects
can depend on the \emph{result} of an effectful operation. Finally, we will
see how this can be used to implement a type-safe and resource-safe protocol
for file management.

\subsection{Dependent States}

Suppose we have a function which reads input from the console, converts it
to an integer, and adds it to a list which is stored in a \texttt{STATE}.
It might look something like the following:

\begin{code}
readInt : { [STATE (List Int), STDIO] } Eff IO ()
readInt = do let x = trim !getStr
             put (cast x :: !get)
\end{code}

\noindent
But what if, instead of a list of integers, we would like to store a
\texttt{Vect}, maintaining the length in the type?

\begin{code}
readInt : { [STATE (Vect n Int), STDIO] } Eff IO ()
readInt = do let x = trim !getStr
             put (cast x :: !get)
\end{code}

\noindent
This will not type check! Although the vector has length \texttt{n} on
entry to \texttt{readInt}, it has length \texttt{S n} on exit. The
\effects{} library allows us to express this as follows:

\begin{code}
readInt : { [STATE (Vect n Int), STDIO] ==>
            [STATE (Vect (S n) Int), STDIO] } Eff IO ()
readInt = do let x = trim !getStr
             putM (cast x :: !get)
\end{code}

\noindent
The notation \texttt{\{ xs ==> xs' \} Eff m a} in a type means that the
operation begins with effects \texttt{xs} available, and ends with effects
\texttt{xs'} available. We have used \texttt{putM} to update the state, where
the \texttt{M} suffix indicates that the \emph{type} is being updated as well
as the value. It has the following type:

\begin{code}
putM : y -> { [STATE x] ==> [STATE y] } Eff m () 
\end{code}

\subsection{Result-dependent Effects}

Often, whether a state is updated could depend on the success or otherwise
of an operation. In our \texttt{readInt} example, we might wish to update
the vector only if the input is a valid integer (i.e. all digits). As a
first attempt, we could try the following, returning a \texttt{Bool} which
indicates success:

\begin{code}
readInt : { [STATE (Vect n Int), STDIO] ==>
            [STATE (Vect (S n) Int), STDIO] } Eff IO Bool
readInt = do let x = trim !getStr
             case all isDigit (unpack x) of
                  False => pure False
                  True => do putM (cast x :: !get)
                             pure True
\end{code}

\noindent
Unfortunately, this will not type check because the vector does not get
extended in both branches of the \texttt{case}!

\begin{code}
MutState.idr:18:19:When elaborating right hand side of Main.case 
block in readInt:
Unifying n and S n would lead to infinite value
\end{code}

\noindent
Clearly, the size of the resulting vector depends on whether or not the
value read from the user was valid. We can express this in the type:

\begin{code}
readInt : { [STATE (Vect n Int), STDIO] ==>
            {ok} if ok then [STATE (Vect (S n) Int), STDIO]
                       else [STATE (Vect n Int), STDIO] }
readInt = do let x = trim !getStr
             case all isDigit (unpack x) of
                  False => with_val False (pure ())
                  True => do putM (cast x :: !get)
                             with_val True (pure ())
\end{code}

\noindent
The notation \texttt{\{ xs ==> {res} xs' \} Eff m a} in a type means that the
effects available are updated from \texttt{xs} to \texttt{xs'}, \emph{and}
the resulting effects \texttt{xs'} may depend on the result of the operation
\texttt{res}, of type \texttt{a}. Here, the resulting effects are computed
from the result \texttt{ok}---if \texttt{True}, the vector is extended, otherwise
it remains the same. We also use \texttt{with\_val} to return a result:

\begin{code}
with_val : (val : a) -> 
           ({ xs ==> xs' val } Eff m ()) -> { xs ==> xs' } Eff m a
\end{code}

\noindent
We cannot use \texttt{pure} here, as before, since \texttt{pure} does not allow
the returned value to update the effects list. The purpose of \texttt{with\_val}
is to update the effects before returning. As a shorthand, we can write

\begin{code}
pureM val
\end{code}

\ldots instead of\ldots

\begin{code}
with_val val (pure ())
\end{code}

\noindent
\ldots so our program is:

\begin{code}
readInt : { [STATE (Vect n Int), STDIO] ==>
            {ok} if ok then [STATE (Vect (S n) Int), STDIO]
                       else [STATE (Vect n Int), STDIO] }
readInt = do let x = trim !getStr
             case all isDigit (unpack x) of
                  False => pureM False
                  True => do putM (cast x :: !get)
                             pureM True
\end{code}

\noindent
When using the function, we will naturally have to check its return value
in order to know what the new set of effects is. For example, to read a set
number of values into a vector, we could write the following:

\begin{code}
readN : (n : Nat) ->
        { [STATE (Vect m Int), STDIO] ==>
          [STATE (Vect (n + m) Int), STDIO] } Eff IO ()
readN Z = pure ()
readN {m} (S k) = case !readInt of
                      True => rewrite plusSuccRightSucc k m in readN k
                      False => readN (S k)
\end{code}

\noindent
The \texttt{case} analysis on the result of \texttt{readInt} means that
we know in each branch whether reading the integer succeeded, and therefore
how many values still need to be read into the vector.

\textbf{Aside:} Only \texttt{case} will work here. We cannot use
\texttt{if/then/else} because the \texttt{then} and \texttt{else} branches must
have the same type. The \texttt{case} construct, however,
abstracts over the value being inspected in the type of each branch.


\subsection{File Management}

\begin{code}[float=h,frame=single, caption={File I/O Effect}, label=eff:fileio]
module Effect.File

FILE_IO : Type -> EFFECT

data OpenFile : Mode -> Type

open  : Handler FileIO e => String -> (m : Mode) -> 
        { [FILE_IO ()] ==> 
          {ok} [FILE_IO (if ok then OpenFile m else ())] } Eff e Bool
close : Handler FileIO e =>
        { [FILE_IO (OpenFile m)] ==> [FILE_IO ()] } Eff e ()

readLine  : Handler FileIO e => 
           { [FILE_IO (OpenFile Read)] } Eff e String 
writeLine : Handler FileIO e => String -> 
           { [FILE_IO (OpenFile Write)] } Eff e ()
eof       : Handler FileIO e => 
           { [FILE_IO (OpenFile Read)] } Eff e Bool 

instance Handler FileIO IO
\end{code}

